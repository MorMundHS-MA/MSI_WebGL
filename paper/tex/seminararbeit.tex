% This is samplepaper.tex, a sample chapter demonstrating the
% LLNCS macro package for Springer Computer Science proceedings;
% Version 2.20 of 2017/10/04
%
\documentclass[runningheads]{llncs}
%
\input{preambel} % Weitere Einstellungen aus einer anderen Datei lesen

% Used for displaying a sample figure. If possible, figure files should
% be included in EPS format.
%
% If you use the hyperref package, please uncomment the following line
% to display URLs in blue roman font according to Springer's eBook style:
% \renewcommand\UrlFont{\color{blue}\rmfamily}

\begin{document}
%
\title{WebGL: 3D-Graphics in the Browser\thanks{Hochschule Mannheim}}
%
%\titlerunning{Abbreviated paper title}
% If the paper title is too long for the running head, you can set
% an abbreviated paper title here
%
\author{Moritz Mundhenke\inst{1} \and
Luca Schilling\inst{1}}
%
\authorrunning{F. Author et al.}
% First names are abbreviated in the running head.
% If there are more than two authors, 'et al.' is used.
%
\institute {Hochschule Mannheim, Paul-Wittsack-Straße 10, 68163 Mannheim, Germany
\email{info@hs-mannheim.de}\\
\url{https://www.hs-mannheim.de/}} 
%
\maketitle              % typeset the header of the contribution
%


\begin{abstract}
The abstract should briefly summarize the contents of the paper in
15--250 words.

\end{abstract}
%
%
%
\input{sections/introduction.tex}

\section{3D Rendering Grundlagen}
In den weiterführenden Abschnitten werden Kenntnisse über grundlegende Begriffe und Konzepte des 3D Renderings benötigt.
Deshalb werden diese in diesem Abschnitt erklärt.
Die benötigten Themen wurden in WebGL: up and running \cite[4-9]{parisi2012webgl} passend zu den folgenden Kapiteln gegliedert, deshalb wurde die Struktur dieser Grundlagen an diese angelehnt.

\subsection{3D Koordinatensystem}
Beim rendern von 2D Elementen wird ein einfach 2D Koordinatensystem verwendet, genauso wie man es aus der Mathematik kennt. Dort gibt es eine x- und eine y-Achse durch welche die Koordinaten der einzelnen Punkte beschrieben werden können.
Dabei ist es üblich, dass die x-Achse horizontal liegt und somit bestimmt wie weit Links oder Rechts der Punkt ist. Die y-Achse hingegen bestimmt wie hoch und tief ein Punkt liegt.
Um nun zum 3D Koordinatensystem zu kommen, wird einfach eine dritte Achse hinzugefügt, die z-Achse. Diese steht beim WebGL Koordinatensystem so gesehen aus dem Bildschirm heraus, das heißt,
 negative Werte geben an wie tief ein Punkt im Bildschirm liegt.\cite[4]{parisi2012webgl} In Abbildung~\ref{fig:3DKoordinatensystem} sieht man exemplarisch ein solches 3D Koordinatensystem.
 \begin{figure}
    \centering
    \includegraphics[width=6cm]{3dkoords.jpg}
    \caption{3D Koordinatensystem \cite{PeterStrohm}} \label{fig:3DKoordinatensystem}
    \end{figure}

\subsection{Gittergewebe, Polygone und Eckpunkte}
\subsection{Materialien, Texturen und Lichter}
\subsection{Transformationen und Matrizen}
\subsection{Kamreas, Perspektiven, Ansichtsfenster und Projektionen}
\subsection{Shader}



\section{WebGL}
\subsection{Funktionsweise}
WebGL basiert auf der OpenGL ES (Embedded Systems) API \cite{parisi2012webgl} einer reduzierten Version der OpenGL API \cite{KhronosGLES}. Reduziert bezieht sich in diesem Falle jedoch primär auf das Entfernen von Abwärtskompatibilität (\zb fixed-function pipeline) sowie das Wegfallen von Funktionen die softwareseitig durch einfachere Funktionen ersetzt werden (\zb Kein direktes rendering von Quads) \cite{DiffGLES}. Daher können selbst state-of-the-art Engines wie die Unreal Engine mit geringen Einschränkungen verwendet werden um WebGL Projekte zu realisieren \cite{UnrealHTML5}\cite{UnrealLimits}. \\
Da es sich bei WebGL somit um eine low-level Grafikapi handelt, ist die Entwicklung von Anwendung direkt in WebGL sehr Zeitaufwendig und setzt fortgeschrittenes Know-how in Grafikprogrammierung voraus. Daher gibt es einige Frameworks und Game Engines die durch Abstraktionen die Entwicklung wesentlich erleichtern \cite{parisi2012webgl}. Eine exemplarische Übersicht über diese wird in den folgenden Kapiteln gezeigt. \\
Für die Verwendung von WebGL in einer Website wird ein HTML5 canvas Element als Interface zwischen der \ac{DOM} und WebGL benötigt. Von diesem canvas Element kann dann ein WebGL Kontext erzeugt werden \cite{parisi2012webgl}. Dieser Kontext wird dann als Interface für alle folgenden WebGL Operationen verwendet. Konkret würde eine Anwendung 3D Modelle und Texturen laden \bzw erzeugen, diese in Puffer laden um diese dann mithilfe von Shaderprogrammen auf den Canvas Framebuffer zu rendern. 
\subsection{Vorteile und Nachteile}
Um die Vor- und Nachteile von WebGL zu erläutern, werden wir WebGL mit den Webtechnologien Canvas2D, CSS 3D transform und SVG vergleichen. Durch die grundlegende, im vorrangehenden Kapitel beschriebene Ähnlichkeit von WebGL und nativen Grafikapis, wie OpenGL und DirectX, werden diese nicht weiter verglichen.

\begin{table}[ht]
    \centering
    \begin{tabular}{|P{2.5cm}|P{2.5cm}|P{2.5cm}|P{2.5cm}|P{2.5cm}|}
        \hline
        Eigenschaften & WebGL & Canvas2D & CSS 3D &  SVG \\ \hline
        3D & \checkmark & \cross & \checkmark & \cross \\ \hline
        inline HTML & \cross (Proposed) & \cross & \checkmark & \cross \\ \hline
        CSS  & \cross & \cross & \checkmark & \checkmark \\ \hline
        VR & \checkmark & \cross & \cross & \cross \\ \hline
    \end{tabular}
    \caption{Übersicht Eigenschaften von Webgrafiktechnologien}
    \label{table:CompWebtech}
\end{table}

\subsubsection*{Canvas2D}
und WebGL sind sich in der Art der Einbindung in Websites sehr ähnlich. Beide sind als Context eines Canvas HTML5 Elements verfügbar. Wie der Name vermuten lässt ist Canvas2D jedoch auf hardware beschleunigte 2D Grafiken optimiert (Primitive 3D Grafik ist durch softwareseitige Emulation möglich). Dadurch ist es leichter zu verwenden, da der Programmierer sich nicht um low-level Details kümmern muss, sondern direkt 2D Primitive wie Rechtecke, Linien und Kreise zeichnen kann.
\subsubsection*{CSS3d}
bezeichnet eine Reihe von Erweiterung von CSS die das manipulieren von \ac{DOM} Objekten im dreidimensionalen Raum erlauben. Ein Beispiel hierfür ist auf \url{https://developer.mozilla.org/en-US/docs/Web/CSS/transform-function/translate3d} verfügbar. Da alle 3D Element Teil der \ac{DOM} sind, ist CSS3D nicht für komplexe Anwendung optimiert. Besonders Updates der 3D Szene mit vielen Änderung würden zu großen Performanceproblemen führen. Der Haupteinsatzzweck für CSS3D ist somit das darstellen von regulären, möglicherweise auch komplexen wie etwa ganze (Sub-)Seiten als 3D Objekt. Beispielhaft hierfür ein 3D Würfel aus Youtube Videos \url{https://threejs.org/examples/#css3d_youtube}.
\subsubsection*{\ac{SVG}}
\ac{SVG} wurden ursprünglich als statische vektoralternative für reguläre Bilder verwendet. Durch die Einführung von mächtigerem JavaScript und CSS ist es jedoch heutzutage möglich auch interaktive Grafiken mit SVG darzustellen. Die verbreitete Datenvisualisierungs Bibliothek d3.js nutzt zum Beispiel SVG zur Anzeige der Grafiken. SVG verwendet, ähnlich wie CSS3D, \ac{DOM} Elemente zur Objektdefinition. Daher ergeben sich die gleichen Performanceprobleme bei größeren \bzw sehr dynamischen Grafiken. Jedoch ermöglicht die Integration mit der \ac{DOM} nicht nur eine einfachere Programmierung, sondern auch die Nutzung von CSS als mächtiges Gestaltungswerkzeug.
\subsection{Frameworks}
\subsection{Game Engines}


\section{Anwendungsbereiche}
In diesem Abschnitt werden einige der aktuellen Anwendungsbereiche von WebGL vorgestellt.
Zu beachten ist, die Abbildungen sind logischwerweiße wieder nur 2D Bilder und können deshalb nicht die vollen Eindrücke der jeweiligen Seiten wiedergeben, dafür ist ein Besuch der Internetseite nötig.
\subsection{Google Maps}
Eine der wohl bekanntesten Beispiele wo WebGL angewendet wird ist Google Maps.
Hierbei profitiert die in Abbildung~\ref{fig:GoogleMaps} gezeigte Satellitenansicht von der dritten Dimension.
Gebäude können dadurch deutlich realitätsnaher dargestellt werden als auf einer einfachen Karte, auf der man nur die Umrisse sieht.
\begin{figure}
    \centering
    \includegraphics[width=6cm]{GoogleMapsExample.jpg}
    \caption{Google Maps Satellitenansicht \cite{GoogleMaps}} \label{fig:GoogleMaps}
    \end{figure}

\subsection{Online Shopping}
Online Shopping ist ein weiters Beispiel das zur heutigen Zeit fast ausschließlich 2 Dimensional stattfindet.
Durcch die fehlende dritte Dimension können Artikel etwas schlechter angesehen werden, die meisten Online Shopping Seiten versuchen dies durch das Anbieten von Fotos aus verschiedenen Perspektiven zu lösen.
Durch WebGL entsteht die Möglichkeit dies in wirklichem 3D darzustellen, als Beispiel hierfür wurde die Seite Hypebeast gewählt, diese stellt zum Beispiel einen Adidas Schuh auf moderne Art vor, zuerst erhält man eine virtuelle Reise auf der man viele Informationen zu dem besagten Schuh erhält, danach kann man diesen Schuh in 3D anschauen, drehen und die verschiedenen Modelle auswählen was man in Abbildung~\ref{fig:HypeBeast} sehen kann.
\begin{figure}
    \centering
    \includegraphics[width=6cm]{Hypebeast.jpg}
    \caption{Hypebeast Ozweego Presentation \cite{hypbeast}} \label{fig:HypeBeast}
    \end{figure}

\subsection{3D Entwicklung und 3D Druck}
Ein weiterer Bereich in dem WebGL bereits sehr stark angewendet wird, ist die 3D Entwicklung und der 3D Druck. In beiden Bereichen wollen Leute 3D Modelle erstellen und diese an andere Leute Verkaufen.
Damit sich der Käufer die Modelle vor dem Kauf bereits komplett anschauen kann, wird auch hier WebGL verwendet. \cite{manuninja}
In Abbildung~\ref{fig:sketchfab} sieht man ein solches 3D Modell, man kann sich das Modell einfach so anschauen, drehen und vergrößern. Außerdem kann man sich dort auch spezielle Details wie die Knochenstruktur des Modells oder
die verwendete Materialien anschauen. Es ist schon fast genauso gut wie wen man das Modell gekauft und in einem 3D Modellierungsprogramm geöffnet hat. Des weiteren gibt es sogar Seiten die das eigentliche 3D Modellieren im Browser ermöglichen wie zum Beispiel SculptGL. \cite{sculptgl}
\begin{figure}
    \centering
    \includegraphics[width=6cm]{sketchfab.jpg}
    \caption{3D Modell Marktplatz \cite{sketchfab}} \label{fig:sketchfab}
    \end{figure}

\subsection{3D Simulationen im Bereich Medizin}
Auch in der Medizin findet WebGL viele Anwendungsbereiche, die Seite BioDigital ermöglicht das Simulieren und Darstellen des Menschlichen Körpers und vielen mehr.
In Abbildung~\ref{fig:BioDigital} sieht man die Simulation eines Menschlichen Körpers der COVID-19 Symptomen aufweißt, diese Simulationen können zur Weiterbildung im Medizinischen Umfeld, zum aufklären von Patienten und vielem mehr verwendet werden.\cite{BioDigital}
\begin{figure}
    \centering
    \includegraphics[width=6cm]{BioDigital.jpg}
    \caption{3D Simulation eines Körpers mit COVID-19 Symptomen \cite{BioDigital}} \label{fig:BioDigital}
    \end{figure}

\section{Ausblick}
\subsection{\acl{WebASM}}
\subsection{WebGPU}
\subsection{Alternativen}
%\include{sections/llncsExample.tex} %Comment out if you want to see how some basic stuff should be written with this template


\bibliography{literatur}{}
\bibliographystyle{plain}
\end{document}

